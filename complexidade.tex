\chapter{Complexidade Computacional}
\label{cha:complexidade}

Até aqui nos ocupamos principalmente do problema da expressivdade de modelos de computação.
Ou seja, o que é possível computar com cada modelo.
Terminamos o último capítulo com um modelo bastante expressivo das Máquinas de Turing.
Vimos que mesmo nesse modelo há problemas que não são computáveis, como o problema da parada.

Neste último capítulo nos voltaremos para outra questão: que problemas computacionais são resolvíveis de maneira eficiente?
Por eficiente entendemos que há algum recurso escasso consumido pelo algoritmo que resolve o problema, por exemplo tempo ou espaço de memória.

\section{Complexidade de Tempo}
\label{sec:tempo}

O {\em tempo de execução} de uma MT $M$ é uma função $f: \mathbb{N} \to \mathbb{N}$ em que $f(n)$ é o número máximo de passos de derivação para uma entrada $\omega$ qualquer de tamanho $n$.

\begin{displaymath}
  TIME(t(n)) = \{A \subseteq \Sigma^* : \textrm{$\exists$ MT simples que decide $A$ em tempo $O(t(n))$}\}
\end{displaymath}

\begin{example}
  $TIME(n)$ é a classe dos problemas resolvíveis em tempos {\em linear} no pior caso.

  $TIME(n^2)$ é a classe dos problemas resolvíveis em tempo {\em quadrático} no pior caso.
\end{example}

\begin{theorem}
  Se $t(n) \geq n$ então toda MT multifita que consome tempo $t(n)$ é equivalente a uma MT simples que consome tempo $O(t^2(n))$.
\end{theorem}
\begin{proof}
  Considere a simulação de uma MT com $k$ fitas que vimos no Teorema \ref{}.

  $M$ varre a fita em tempo $O(n)$ para obter as informação necessárias para o próximo passo.

  Para executar um passo $M$ no pior precisamos abrir um espaço em branco na fita e para isso deslocamos todo conteúdo uma posição para a direita.
  Nesse caso como o tamanho máximo da fita é $O(t(n))$, precisaríamos de $O(t(n))$ passos para esse deslocamento.

  Assim, o tempo total de excecução é $t(n).O(t(n)) + O(n)$.
  Se $t(n) \geq n$ então $t(n).O(t(n)) + O(n) = O(t^2(n))$.
\end{proof}

O tempo de execução de uma MT não-determinística $N$ é uma função $f: \mathbb{N} \to \mathbb{N}$ em que $f(n)$ é o número máximo de passos de {\em alguma} derivação de $N$ para a entrada $\omega$ de tamanho $n$.

\begin{multicols}{2}
\centering
Determinístico

\vspace{1cm}

\begin{tikzpicture}[node distance=2cm,auto,>=latex,initial text=]
  \draw [|-|] (-2,0) -- node[left]{$f(n)$} (-2,-8);
  \node[circle, draw] (q0) {};
  \node[circle, draw] (q1) at (0, -2) {};
  \node[circle, draw] (q2) at (0, -4) {};
  \node[circle, draw] (qn) at (0, -8) {};
  \path[->] (q0) edge (q1);
  \path[->] (q1) edge (q2);
  \path[-, dashed] (q2) edge (qn);
\end{tikzpicture}


\columnbreak
\centering
Não Determinístico

\vspace{1cm}

\begin{tikzpicture}[node distance=2cm,auto,>=latex,initial text=]
  \node[circle, draw] (q0) {};
  \node[circle, draw] (q11) at (-2, -2) {};
  \node[circle, draw] (q12) at (0,  -2) {};
  \node[circle, draw] (q13) at (2,  -2) {};
  \node[circle, draw] (q21) at (-4, -4) {};
  \node[circle, draw] (q22) at (-2, -4) {};
  \node[circle, draw] (q23) at (0,  -4) {};
  \node[circle, draw] (q24) at (2,  -4) {};
  \node[circle, draw] (q25) at (4,  -4) {};
  \node[circle, draw] (qn1) at (-4, -8) {};
  \node[circle, draw] (qn2) at (-2, -8) {};
  \node[circle, draw] (qn3) at (0,  -8) {};
  \node[circle, draw] (qn4) at (2,  -8) {};
  \node[circle, draw] (qn5) at (4,  -8) {};
  \path[->] (q0) edge (q11);
  \path[->] (q0) edge (q12);
  \path[->] (q0) edge (q13);
  \path[->] (q11) edge (q21);
  \path[->] (q11) edge (q22);
  \path[->] (q12) edge (q23);
  \path[->] (q13) edge (q24);
  \path[->] (q13) edge (q25);
  \path[-, dashed] (q21) edge (qn1);
  \path[-, dashed] (q22) edge (qn2);
  \path[-, dashed] (q23) edge (qn3);
  \path[-, dashed] (q24) edge (qn4);
  \path[-, dashed] (q25) edge (qn5);
\end{tikzpicture}

\end{multicols}


\begin{theorem}
  Se $t(n) \geq n$ então toda MT não-determinística que consome tempo $t(n)$ é equivalente a uma MT simples que consome tempo $2^{O(t(n))}$.
\end{theorem}
\begin{proof}
  Vimos no Teorema \ref{} como simular uma MT não-determinística $N$ usando uma MT com 3 fitas usando uma busca em largura.

  Seja $b$ o número máximo de ramificações de na excecuçaõ $N$.
  O número total de nós da árvore é $O(b^{t(n)})$ e a excecução de cada nó toma tempo $O(t(n))$ no pior caso.

  Assim, o tempo total de excecução dessa simulação é $O(t(n).b^{t(n)}) = 2^{O(t(n))}$ se $t(n) > n$.

  Por fim, essa MT de três fitas pode ser simulada por uma MT simples que consome tempo $2^{O(t^2(n))} = 2^{2O(t(n))} = 2^{O(t(n))}$.
\end{proof}

\begin{displaymath}
  NTIME(t(n)) = \{A \subseteq \Sigma^* : \textrm{$\exists$ MT não-det. que decide $A$ em tempo $O(t(n))$}\}
\end{displaymath}

Vamos definir duas classes de complexidade de tempo.
A classe $P$ contém todas as linguagens decidíveis por MT simples em tempo polinomial e a classe $NP$ que contém todas as linguágens decidíveis por MTs não-determinísticas em tempo polinomial:
\begin{displaymath}
  P = \bigcup_k TIME(n^k)
\end{displaymath}

\begin{displaymath}
  NP = \bigcup_k NTIME(n^k)
\end{displaymath}

É evidente que toda linguagem em $P$ pertence a $NP$.
Ou seja, $P \subseteq NP$.
Não sabemos, porém, se é verdade que $NP \subseteq P$.
Em outra palavras, se existem soluções polinomiais em MTs simples para os problemas em que possuem solução em MTs não-determinísticas.
Esse é o principal problema em aberto na computação.

Uma forma alternativa de apresentar a classe de problemas NP é por meio de um {\em oráculo}.
Um {\em oráculo} (ou verificador) para uma linguagem $A$ é um algotimo $V$ tal que:
\begin{displaymath}
  A = \{\omega : V \textrm{ aceita $\langle \omega, o \rangle$ para alguma string $o$}\}
\end{displaymath}

\begin{example}
  Seja $L = \{p_1, \dots, p_n, \bar{p_1} \dots \bar{p_n}\}$ uma alfabeto.
  Uma {\em cláusula} sobre $L$ é uma string $c \in L^*$ e uma {\em fórmula} é uma string $f \in (L\cup\{;\})^*$.
  Uma {\em valoração} é uma função $v : L \to \{0,1\}$ tal que $v(p) = 1$ sse $v(\bar{p}) = 0$.
  Uma valoração $v$ {\em satisfaz uma cláusula} $c$ se $v(l) = 1$ para {\em algum} $l$ em $c$ e $v$ {\em satisfaz uma fórmula} $f = c_1;c_2;\dots;c_m$ se ele satisfaz {\em todas} as cláusulas $c_1, \dots, c_n$.

  Definimos o {\em problema da satisfatibilidade} da seguinte forma:
  \begin{displaymath}
    SAT = \{f \in (L \cup \{;\})^*: \textrm{existe $v$ que satisfaz $f$}\}
  \end{displaymath}

  Uma valoração pode ser descrita como uma string $o \in \{0,1\}^*$.
  (Por exemplo, a string $101$ indica que $v(p_1) = 1$, $v(p_2) = 0$ e $v(p_3) = 1$).

  É fácil construir uma MT $V$ que recebe uma fórmula $f \in (L \cup \{;\})^*$ e um string $o \in \{0,1\}^*$ e aceita se a valoração $v$ representada por $o$ satisfaz $f$ e rejeita caso contrário.
  Essa verificação pode ser feita em tempo polinomial em relação a $|f|$.
  Note que podemos descrever o problema SAT da seguinte forma:

  \begin{displaymath}
    SAT = \{f \in (L \cup \{;\})^*: \textrm{$V$ aceita $\langle f, o \rangle$ para algum $o \in \{0,1\}^*$}\}
  \end{displaymath}

  Dizemos, portanto, que $V$ é um {\em verificador polinomial} para SAT.
\end{example}

\begin{theorem}
  Uma linguagem $A \in NP$ sse exsite um verificador polinomial para $A$.
\end{theorem}
\begin{proof}
  Se $A \in NP$ então, por definição, existe uma MT não-determinística $N$ que decide $A$ em tempo polinomial.
  Considere uma string $\omega$ qualquer.
  Se $\omega \in A$ então $N$ aceita, senão rejeita.
  De qualquer forma existe um ramo da excecução de $N$ que termina em menos de $O(n^k)$ passos.
  Seja $o$ a codificação desse ramo (a string que indica a cada passo qual o caminho que foi seguido).
  Simulando $N$ como uma MT com três fitas, e colocando $o$ na terceira, decidimos se $\omega$ é aceito ou não em tempo polinomial.

  Agora considere o outro lado.
  Seja $V$ ium verificador para $A$ que decide se a entrada é aceita em tempo $O(n^k)$.
  Escolhemos não deterministicamente uma string $o$ com tamanho máximo $n^k$.
  Em cada ramo e excecutamos $V$ sobre $\langle \omega, o \rangle$ para um $o$ distinto e aceitamos $\omega$ se $V$ aceitar $\langle \omega, o \rangle$ para algum $o$.
  Se nenhum ramo $V$ aceitar a entrada então rejeitamos $\omega$.
\end{proof}

\section{NP-completude}
\label{sec:np-completude}

Na última seção definimos as classes $P$ e $NP$ e mencionamos que a pergunta se $P \stackrel{?}{=} NP$ é um problema em aberto na computação.
O que faremos então será tentar classificar que problemas são mais ``fáceis'' ou mais ``difíceis'' do que outros.

Dizemos que uma função $f : \Sigma^* \to Sigma^*$ é {\em computável em tempo polinomial} se existe um polinômio $p$ e uma MT que ao receber $\omega \in \Sigma^*$ para depois de $p(|\omega|)$ passos e devolve $f(\omega)$.

Uma linguagem $A$ é {\em polinomialmente redutível} a $B$ (escrevemos $A \leq_p B$) se existe $f: \Sigma^* \to \Sigma^*$ que seja computável em tempo polinomial e tal que $\omega \in A$ sse $f(\omega) \in B$.

O teorema a seguir mostra que a redutibilidade polinonimal preserva o pertencimento na classe $P$:

\begin{theorem}
  Se $A \leq_P B$ e $B \in P$ então $A \in P$.
\end{theorem}
\begin{proof}
  Seja $M$ uma MT que decide $B$ em tempo polinomial e seja $f$ a redução polinomial de $A$ em $B$.
  Construímos uma MT $N$ da seguinte forma: $N$ recebe $\omega$ e computa $f(\omega)$ então roda $M$ sobre $f(\omega)$.

  Pela definição de $f$, $M$ aceita $f(\omega)$ sse $\omega \in A$ e, portanto, $N$ aceita $\omega$.
  Além disso, $N$ é polinomial pois cada passo é polinomial e polinômios são fechados por composição.
\end{proof}

\begin{example}
  \begin{displaymath}
    3SAT = \{f \in SAT : \textrm{cada clásula de $f$ tem tamanho exatamente 3}\}
  \end{displaymath}

  Vamos mostrar que $SAT \leq_P 3SAT$.

  A transformação vai substituir cada cláusula $c_i = l_1 \dots l_n$ de cada fórmula $f = c_1;c_2; \dots; c_m$ pela seguinte sequência de cláusulas: $l_1l_2m_1;\overline{m_1}l_3m_2;\overline{m_2}l_4m_3; \dots ; \overline{m_{n-3}}l_{n-1}l_n$.
  Essa transformação é claramente polinomial e é possível mostrar que $f \in SAT$ sse essa nova fórmula também for satisfatível.
\end{example}

Uma linguagem $A$ é {\em NP-completa} se:
\begin{itemize}
\item $A \in NP$ e
\item para todo $B \in NP$ temos que $B \leq_P A$
\end{itemize}

Os seguintes são corolários da definição de NP-completude:

\begin{corollary}
  Seja $A$ uma linguagem NP-completa, se $A \in P$ entçao $P = NP$.
\end{corollary}

\begin{corollary}
  Se $A$ é NP-completa e $A \leq_P B$ então $B$ também é NP-completa.
\end{corollary}

Ou seja, intuitivamente as linguagens NP-completas são as mais difíceis dentro da classe NP.
Além disso, se conhecemos uma linguagem NP-completa, então podemos inferir que outras linguagens também o são por redução polinomial.

Resta mostrar que pelo menos uma linguagem é NP-completa.

\begin{theorem}[Cook-Levin]
A linguagem SAT é NP-completa.
\end{theorem}
\begin{proof}
  Mostramos na última seção que $SAT \in NP$.
  Temos que mostrar que $B \neq_P SAT$ para todo $B \in NP$.
  Se $B \in NP$ então existe uma MT não-determinística $N$ que decide $B$ em tempo polinomial $n^k$.

  Um {\em tableau} para $N$ sobre a entrada $\omega$ é uma tabela $n^k \times n^k$ cujas linhas são configurações de um ramo de $N$ com entrada $\omega$.
  Assim, a primeira linha contém a configuração inicial e deve haver um tableau que contém uma configuração de aceitação para cada $\omega \in B$.

  % diagrama

  Vamos representar o tableuau como um fórmula $f$ que é satisfatível sse existe um tableau que aceita $\omega$.

  Seja $C = Q \cup \Gamma \cup \{\#\}$, temos uma variável $x_{i,j,s}$ para cada $i,j \in \{1, \dots, n^k\}$ e cada $s \in C$.
  A ideia é que uma valoração $v$ satisfaz $x_{i,j,s}$ se a célula $\langle i, j \rangle$ no tableau contém o símbolo $s$.
  Projetaremos a fórmula $f$ de modo que uma valoração que satisfaz $f$ corresponde a um tableau que reconhece $\omega$.

  \begin{displaymath}
    f_c = x_{1,1,s_1}x_{1,1,s_2} \dots x_{1,1,s_n}; \overline{x_{1,1,s_1}x_{1,1,s_2}}; \overline{x_{1,1,s_1}x_{1,1,s_3}} \dots; x_{1,2,s_1}x_{1,2,s_2} \dots
  \end{displaymath}

  A fórmula $f_c \in SAT$ sse cada célula contém exatamente um símbolo.

  Escrevemos a fórmula $f_i$ de forma que $f_i \in SAT$ sse a primeira linha do tableau contém a configuração inicial de $N$.

  \begin{displaymath}
    f_a = x_{1,1,q_a}x_{1,2,q_a} \dots x_{n^k,n^k, q_a}
  \end{displaymath}

  A fórmula $f_a \in SAT$ sse alguma linha é uma configuração de aceitação.

  Uma {\em janela} $2 \times 3$ no tableua é {\em legal} se não viola as ações especificadas pela função de transição de $N$ (Exemplo \ref{ex:janela}).
  Escrevemos $f_m$ como a conjunção de todas as janelas legais.
  Ou seja, $f_m$ é tal que $f_m \in SAT$ sse a configuração da linha $i$ segue da configurção da linha $i-1$ em $N$.

  Assim, a fórmula $f = f_c;f_i;f_a;f_m \in SAT$ sse $\omega \in B$ para algum $B \in NP$.
\end{proof}

\begin{example}
  \label{ex:janela}
  Considere que $\Delta(q_1, b) = \{\langle q_2, c, E\rangle, \langle q_2, a, D \rangle\}$, as seguintes janelas são legais:

  \begin{displaymath}
    \begin{array}{|c|c|c|}
      \hline
      a & q_1 & b \\
      \hline
      a & a & q_2 \\
      \hline
    \end{array}
  \end{displaymath}

    \begin{displaymath}
    \begin{array}{|c|c|c|}
      \hline
      a & q_1 & b \\
      \hline
      q_2 & a & c \\
      \hline
    \end{array}
  \end{displaymath}

\end{example}

\begin{corollary}
  3SAT é NP-completa
\end{corollary}

%\section{Complexidade de Espaço}
%\label{sec:espaco}
