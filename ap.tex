\chapter{Linguagens Livres de Contexto}
\label{cha:ap}

Estudamos até exaustivamente uma classe de linguagens, as regulares.
Apresentamos essa classe de maneira declarativa por meio de expressões regulares e imperativa com dois modelos de computação: autômatos finitos determinísticos e não-determinísticos que vimos serem equivalentes.
No fim do capítulo mostramos que nem toda linguagem é regular.

Neste capítulo estudaremos uma classe de linguagens mais completa, as linguagens livres de contexto.
Como as linguagens regulares, apresentaremos tais linguagens de maneira declarativa por meio de gramáticas livres de contexto e imperativa por meio dos autômatos com pilha.

\section{Introdução}
\label{sec:llc}

Uma {\em Gramática Livre de Contexto} (GLC) é uma 4-upla $\langle V, \Sigma, R, S \rangle$ em que:
\begin{itemize}
\item[] $V$ é um conjunto finito cujos elementos são chamados {\em variáveis},
\item[] $\Sigma$ é um conjunto finito disjunto de $V$ (i.e. $\Sigma \cap V \neq \emptyset$) cujos elementos são chamados {\em terminais},
\item[] $R$ é um conjunto finito de {\em regras} e cada regra é da forma $v_1 \to v_2 \dots v_n$ onde $v_1 \in V$ e $v_i \in V \cup \Sigma$ para $i = 2, \dots, n$ e
\item[] $S \in V$ é uma variável chamada {\em inicial}.   
\end{itemize}

Se $u$, $v$ e $w$ strings sobre o alfabeto $V \cup \Sigma$ e $A \to w$ é uma regra da gramática, dizemos que $uAv$ {\em origina} $uwv$ (escrevemos $uAv \Rightarrow uwv$).
Dizemos que $u$ {\em deriva} $v$ (escrevemos $u \Rightarrow^* v$) se $u = v$ ou existe uma sequência $u_1, \dots, u_k$ para $k \geq 0$ em que:
\begin{displaymath}
  u \Rightarrow u_1 \Rightarrow \dots \Rightarrow u_k \Rightarrow v
\end{displaymath}

A {\em linguagem associada à gramatica} $G = \langle V, \Sigma, R, S \rangle$, ou simplesmente a linguagem de $G$ é $\{\omega \in \Sigma^* : S \Rightarrow^* \omega \}$.

Se uma linguagem $A$ possui uma gramatica livre de contexto associada a ele então $A$ é chamada {\em linguagem livre de contexto}.

\begin{example}
Considere a seguinte GLC $G = \langle V, \Sigma, R, S \rangle$:
\begin{itemize}
\item $V = \{S\}$
\item $\Sigma = \{0, 1\}$
\item $R = \{S \to \varepsilon, S \to 0S1\}$
\end{itemize}

$01$ pertence a linguagem dessa gramática:


\begin{eqnarray*}
  S & \Rightarrow & 0S1 \\
    & \Rightarrow & 0 \varepsilon 1 = 01  
\end{eqnarray*}

$000111$ pertence a linguagem dessa gramática:


\begin{eqnarray*}
  S & \Rightarrow & 0S1\\
    & \Rightarrow & 00S11\\
    & \Rightarrow & 000S111\\
    & \Rightarrow & 000\varepsilon 111\\
    & \Rightarrow & 000111  
\end{eqnarray*}

Não é difícil notar que a linguagem dessa gramática é:
\begin{displaymath}
  \{0^n 1^n : n \geq 0 \}
\end{displaymath}
\end{example}

Para apresentar as próximas gramáticas, usaremos a seguinte abreviação:
\begin{displaymath}
  \{A \to w, A \to u, A \to v\}
\end{displaymath}

Será substituído simplesmente por:

\begin{displaymath}
  A \to w | u | v
\end{displaymath}

\begin{example}
$G = \langle \{S\}, \{0,1\}, R, S\rangle$ em que $R$ é:
\begin{displaymath}
  S \to SS | 0 | 1
\end{displaymath}
 
$01$ pertence à linguagem de $G$:

\begin{eqnarray*}
  S & \Rightarrow & SS\\
    & \Rightarrow & 0S \\
    & \Rightarrow & 01
\end{eqnarray*}

$0101$ pertence à linguagem de $G$:

\begin{eqnarray*}
  S & \Rightarrow & SS \\
    & \Rightarrow & 0S \\
    & \Rightarrow & 0SS \\
    & \Rightarrow & 0S0 \\
    & \Rightarrow & 010
\end{eqnarray*}
\end{example}


\begin{example}
\begin{displaymath}
G = \langle \{S\}, \{0, 1, \star, \abxcup, \epsilon, \o\}, R, S\rangle  
\end{displaymath}
\begin{displaymath}
  S \to 0 | 1 | \epsilon | \o | SS | S \abxcup S | S^\star
\end{displaymath}

Vamos mostrar que $10 \abxcup 1^\star \in L(G)$

\begin{eqnarray*}
  S & \Rightarrow & S \abxcup S \\
    & \Rightarrow & SS \abxcup S \\
    & \Rightarrow & SS \abxcup S^\star\\
    & \Rightarrow & 1S \abxcup S^\star\\
    & \Rightarrow & 10 \abxcup S^\star\\
    & \Rightarrow & 10 \abxcup 1^\star
\end{eqnarray*}
\end{example}

\begin{example}
  \begin{displaymath}
    G = \langle V, \Sigma, R, Expr \rangle
  \end{displaymath}
  \begin{itemize}
  \item $V = \{Expr, Termo, Fator\}$
  \item $\Sigma = \{a, +, \cdot, (,)\}$
  \end{itemize}

  
  \begin{eqnarray*}
    Expr &\to& Expr  +  Termo |  Termo \\
    Termo &\to& Termo \cdot Fator |  Fator \\
    Fator &\to& ( Expr ) | a \\
  \end{eqnarray*}

Vamos mostrar que $a + a \cdot a \in L(G)$.

\begin{eqnarray*}
  Expr  & \Rightarrow &  Expr  +  Termo  \\
  & \Rightarrow &  Expr  +  Termo  \cdot  Fator \\
  & \Rightarrow &  Termo  +  Termo  \cdot  Fator \\
  & \Rightarrow &  Fator  +  Fator  \cdot  Fator \\
  & \Rightarrow & a + a \cdot a\\
\end{eqnarray*}
\end{example}

Podemos representar a derivação do último exemplo por meio de uma {\em árvore sintática}:

\begin{center}
  \begin{tikzpicture}[level distance=2cm,sibling distance=.5cm]
    \Tree[.$Expr$
      \edge[-]node{}; [.$Expr$
        \edge[-]node{}; [.$Termo$ 
          \edge[-]node{}; [.$Fator$
            \edge[-]node{}; [.$a$ ]   
          ] 
        ]
      ]
      \edge[-]node{}; [.$+$ ]
      \edge[-]node{}; [.$Termo$ 
        \edge[-]node{}; [.$Termo$
          \edge[-]node{}; [.$Fator$
            \edge[-]node{}; [.$a$ ]
          ]
        ]
        \edge[-]node{}; [.$\cdot$ ]
        \edge[-]node{}; [.$Fator$ 
          \edge[-]node{}; [.$a$ ]
        ]
      ]
    ]
  \end{tikzpicture}
\end{center}


Note que uma mesma string pode ser derivada de uma mesma gramática por diferentes árvores sintáticas.
Esse fenômeno é chamado {\em ambiguidade}.

Uma derivação de uma string $\omega$ em uma gramática $G$ é uma {\em derivação mais a esquerda} se a cada passo a variável remanescente mais a esquerda é aquela que será substituída no próximo passo.
Uma string é {\em derivada de maneira ambígua} na gramática $G$ se ela tem mais de uma derivação à esquerda.
Uma GLC é dita {\em ambígua} se ela gera alguma string de maneira ambígua.


\begin{example}
\begin{displaymath}
  G = \langle \{S\}, \{+, \cdot, a\}, R, S \rangle
\end{displaymath}

\begin{displaymath}
  S \to S + S | S \cdot S | a
\end{displaymath}

Vamos derivar a esquerda a expressão $a + a \cdot a$:

\begin{eqnarray*}
  S & \Rightarrow & S + S\\
    & \Rightarrow & a + S\\
    & \Rightarrow & a + S \cdot S\\
    & \Rightarrow & a + a \cdot a
\end{eqnarray*}

Alternativamente podemos derivar a mesma expressão à esquerda da seguinte maneira:

\begin{eqnarray*}
  S & \Rightarrow & S \cdot S\\
    & \Rightarrow & S + S \cdot S\\
    & \Rightarrow & a + S \cdot S\\
    & \Rightarrow & a + a \cdot a
\end{eqnarray*}

Essas derivações são representadas pelas seguintes árvores sintáticas.


\begin{multicols}{2}
\begin{center}
  \begin{tikzpicture}[level distance=2cm,sibling distance=.5cm]
    \Tree[.$S$
      \edge[-]node{}; [.$S$
        \edge[-]node{}; [.$a$ ]   
      ]
      \edge[-]node{}; [.$+$ ]
      \edge[-]node{}; [.$S$ 
        \edge[-]node{}; [.$S$
          \edge[-]node{}; [.$a$ ]
        ]
        \edge[-]node{}; [.$\cdot$ ]
        \edge[-]node{}; [.$S$ 
          \edge[-]node{}; [.$a$ ]
        ]
      ]
    ]
  \end{tikzpicture}
\end{center}
  \columnbreak
\begin{center}
  \begin{tikzpicture}[level distance=2cm,sibling distance=.5cm]
    \Tree[.$S$
      \edge[-]node{}; [.$S$
        \edge[-]node{}; [.$S$
          \edge[-]node{}; [.$a$ ]
        ]
        \edge[-]node{}; [.$+$ ]
        \edge[-]node{}; [.$S$
          \edge[-]node{}; [.$a$ ]
        ]
      ]
      \edge[-]node{}; [.$\cdot$ ]
      \edge[-]node{}; [.$S$ 
        \edge[-]node{}; [.$a$ ]
      ]
    ]
  \end{tikzpicture}
\end{center}
\end{multicols}

  
\end{example}

Uma GLC está na {\em Forma Normal de Chomsky} (FNC) se toda regra é de uma das seguintes formas:
\begin{itemize}
\item[] $S \to \varepsilon$
\item[] $A \to BC|a$ 
\end{itemize}

Onde $a \in \Sigma$, $A \in V$ e $B,C \in V - \{S\}$.

\begin{theorem}
  Toda linguagem livre de contexto é gerada por uma GLC na FNC.  
\end{theorem}
\begin{proof}
  Essa prova é construtiva.

\begin{enumerate}
\item Criamos um novo estado $S_0$ e uma regra $S_0 \to S$
\item Removemos cada regra da forma $A \to \varepsilon$ e criamos uma nova regra para cada ocorrência de $A$ a direita de uma regra em que $A$ não ocorre (por exemplo $R \to uAvAw$ gera três regras $R \to uvAw$, $R\to uAvw$ e $R \to uvw$).
\item Removemos todas as regras da forma $A \to B$ e criamos uma regra $A \to u$ para cada ocorrência de $B \to u$.
\item Substituímos $A \to u_1u_2 \dots u_k$ onde $k > 2$ e $u_i \in V \cup \Sigma$ por $A \to u_1A_1$, $A \to u_2A_2$, $\dots$, $A_{k-2} \to u_{k-1}u_k$.
\item Substituímos $A \to cB$ (e $A \to Bc$) por $A \to BC$ (ou $A \to CB$) e $C \to c$  
\end{enumerate}
\end{proof}

\begin{example}
 Considere por exemplo a seguinte GLC:
  \begin{eqnarray*}
    S &\to& ASA | aB\\
    A &\to& B | S\\
    B &\to& b | \varepsilon
  \end{eqnarray*}

  Aplicando o primeiro passo da construção obetmos:
\begin{eqnarray*}
    S_0 &\to& S\\
    S &\to& ASA | aB\\
    A &\to& B | S\\
    B &\to& b | \varepsilon
  \end{eqnarray*}

  Aplicando o segundo passo primeiro removemos $B \to \varepsilon$:
  \begin{eqnarray*}
    S_0 &\to& S\\
    S &\to& ASA | aB | a\\
    A &\to& B | S | \varepsilon\\
    B &\to& b
  \end{eqnarray*}

  Em seguida removemos $A \to \varepsilon$
  \begin{eqnarray*}
    S_0 &\to& S\\
    S &\to& ASA | AS | SA | aB | a\\
    A &\to& B | S\\
    B &\to& b
  \end{eqnarray*}


  Aplicando terceiro passo ficamos com o seguinte:
  \begin{eqnarray*}
    S_0 &\to& ASA | AS | SA | aB | a\\
    S &\to& ASA | AS | SA | aB | a\\
    A &\to& b | ASA | AS | SA | aB | a\\
    B &\to& b
  \end{eqnarray*}

  O quarto passo consiste em substituir as sequência de mais de duas variáveis não terminais:
  \begin{eqnarray*}
    S_0 &\to& AA_1 | SA | AS | SA| S | aB | a\\
    S &\to& AA_1 | AS | SA| S | aB | a\\
    A &\to& b | AA_1 | AS | SA| S | aB | a\\
    A_1 &\to& SA\\
    B &\to& b
  \end{eqnarray*}

  Para concluir substituímos os símbolos terminais em regras com um símbolo não terminal:
  \begin{eqnarray*}
    S_0 &\to& AA_1 | SA | AS | SA | UB | a\\
    S &\to& AA_1 | AS | SA | UB | a\\
    A &\to& b | AA_1 | AS | SA | UB | a\\
    A_1 &\to& SA\\
    B &\to& b\\
    U &\to& a
  \end{eqnarray*}


\end{example}


\section{Autômatos de Pilha}
\label{sec:ap}

Em um dos exemplos da seção anterior vimos que a linguagem não-regular $\{0^n1^n : n \geq 0\}$ é livre de contexto.
Portanto, os autômatos finitos não são adequados para reconhecer LLC.
Nesta seção veremos um novo modelo de computação chamado autômato com pilha  e mais para frente mostraremos sua relação íntima com as LLCs.

Um {\em autômato com pilha} (AP) é uma 6-upla $\langle Q, \Sigma, \Gamma, \delta, q_0, F \rangle$ onde:
\begin{itemize}
\item $Q$ é um conjunto finito cujos elementos são chamados {\em estados},
\item $\Sigma$ é um alfabeto chamado {\em alfabeto de entrada},
\item $\Gamma$ é um alfabeto chamado {\em alfabeto da pilha},
\item $\delta: Q \times (\Sigma \cup \{\varepsilon\}) \times (\Gamma \cup \{\varepsilon) \to 2^{Q \times (\Gamma \cup \{\varepsilon\})}$ é a {\em função de transição},
\item $q_0 \in Q$ é o {\em estado inicial} e
\item $F \subseteq Q$ é o conjunto dos {\em estados finais}.
\end{itemize}

Um AP $M = \langle Q, \Sigma, \Gamma, \delta, q_0, F \rangle$ {\em aceita} uma string $\omega$ se $\omega = \omega_1 \omega_2 \dots \omega_n$ onde $\omega_i \in \Sigma \cup \{\varepsilon\}$ e existe uma sequência de estados $r_0, r_1, \dots, r_m \in Q$ e uma sequência de strings $s_0, s_1, \dots, s_m \in \Gamma^*$ tal que:
\begin{enumerate}
\item $r_0 = q_0$ e $s_0 = \varepsilon$ (a pilha começa vazia),
\item $\langle r_{i+1}, b \rangle \in \delta(r_i, \omega_{i+1}, a)$ onde $s_i = at$ e $s_{i+1} = bt$ para $t \in \Gamma^*$ (lê um símbolo, vai para o próximo estado e atualiza a pilha) e
\item $r_m \in F$ (termina em um estado final).
\end{enumerate}

A cada passo o AP lê um símbolo $\omega_{i+1} \in \Sigma \cup \{\varepsilon\}$, desempilha um símbolo $a \in \Gamma \cup \{\varepsilon\}$ da pilha, empilha outro $b \in \Gamma \cup \{\varepsilon\}$ e vai para o novo estado $r_{i+1}$.

\begin{example}

  \begin{eqnarray*}
    M & = & \langle Q, \Sigma, \Gamma, \delta, q_1, F \rangle\\
    Q & = & \{q_1, q_2, q_3, q_4\}\\
    \Sigma & = & \{0, 1\}\\
    \Gamma & = & \{0, \$\}\\
    F & = & \{q_1, q_4\}
  \end{eqnarray*}
  

  $\delta$ é dado pela seguinte tabela:
  
  \begin{tabular}{c|ccc|ccc|ccc|}
    &\multicolumn{3}{|c|}{$0$} & \multicolumn{3}{|c|}{$1$} & \multicolumn{3}{|c|}{$\varepsilon$}\\
    \hline
    &$0$ & $\$$ & $\varepsilon$ & $0$ & $\$$ & $\varepsilon$ & $0$ & $\$$ & $\varepsilon$\\
    \hline
    $q_1$ &&&&&&&&&$\{(q_2,\$)\}$\\
    $q_2$ &&&$\{(q_2,0)\}$&$\{(q_3, \varepsilon)\}$&&&&&\\
    $q_3$ &&&&$\{(q_3,\varepsilon)\}$&&&&$\{(q_4, \varepsilon)\}$&\\
    $q_4$ &&&&&&&&&\\
  \end{tabular}

  Na tabela omitimos as células que deveriam ser preenchidas por $\emptyset$ deixando-as vazias.

  $M$ reconhece a string $01$:
  \begin{enumerate}
  \item Começo no estado $q_1$, emppilho $\$$ e vou para $q_2$ (pilha: $\$$).
  \item Leio $0$, empilho $0$ e fico em $q_2$ (pilha: $0\$$).
  \item Leio $1$, desempilho $0$ e vou para $q_3$ (pilha: $\$$).
  \item Desempilho $\$$ e vou para $q_4 \in F$ (pilha: $\varepsilon$). 
  \end{enumerate}
\end{example}

Podemos representar um AP usando um diagrama de estados.
O diagrama de estados de um AP é como um diagrama de AFNs, mas em cada transição, além do símbolo a ser lido, indicamos os símbolos a serem desempilhados e empilhados (exemplo $a \to b$ indica que deve-se desempilhar $a$ e empilhar $b$).


\begin{example}
Vamos ilustrar o autômato do último exemplo:
\begin{center}
\begin{tikzpicture}[node distance=2cm,auto,>=latex]
\tikzset{initial text={}}
\node[state, initial, accepting] (q1) {$q_1$};
\node[state] (q2) at (4,0) {$q_2$};
\node[state] (q3) at (4,-3) {$q_3$};
\node[state, accepting] (q4) at (0,-3) {$q_4$};
\path[->] (q2) edge[loop above] node {$0$, $\varepsilon \to 0$} (q2);
\path[->] (q3) edge[loop below] node {$1$, $0 \to \varepsilon$} (q3);
\path[->] (q1) edge node {$\varepsilon$, $\varepsilon \to \$$} (q2);
\path[->] (q2) edge node {$1$, $0 \to \varepsilon$} (q3);
\path[->] (q3) edge node {$\varepsilon$, $\$ \to \varepsilon$} (q4);
\end{tikzpicture}
\end{center}
A pilha garante que será reconhecida a mesma quantidade de 0s e de 1s.
Portanto $L(G) = \{0^n1^n : n \geq 0\}$
\end{example}

\begin{example}
\begin{center}
\begin{tikzpicture}[node distance=2cm,auto,>=latex]
  \tikzset{initial text={}}
  \node[state, initial, accepting] (q1) {$q_1$};
\node[state] (q2) at (4,0) {$q_2$};
\node[state] (q3) at (4,-3) {$q_3$};
\node[state, accepting] (q4) at (0,-3) {$q_4$};
\path[->] (q2) edge[loop above] node[align=center]{$0,\varepsilon \to 0$\\$1, \varepsilon \to 1$} (q2);
\path[->] (q3) edge[loop below] node[align=center]{$0$, $0 \to \varepsilon$\\$1$, $1 \to \varepsilon$} (q3);
\path[->] (q1) edge node {$\varepsilon$, $\varepsilon \to \$$} (q2);
\path[->] (q2) edge node {$\varepsilon$, $\varepsilon \to \varepsilon$} (q3);
\path[->] (q3) edge node {$\varepsilon$, $\$ \to \varepsilon$} (q4);
\end{tikzpicture}
\end{center}  

Procure verificar com alguns exemplos que $L(G) = \{\omega \omega^R: \omega \in \{0,1\}^*\}$ aonde $\omega^R$ é $\omega$ escrito de trás para frente.
\end{example}

\begin{example}
  \begin{center}
\begin{tikzpicture}[node distance=2cm,auto,>=latex]
\tikzset{initial text={}}
  \node[state, initial] (q1) {$q_1$};
\node[state] (q2) at (3,0) {$q_2$};
\node[state] (q3) at (6,2) {$q_3$};
\node[state, accepting] (q4) at (9,2) {$q_4$};
\node[state] (q5) at (6,-2) {$q_5$};
\node[state] (q6) at (9,-2) {$q_6$};
\node[state, accepting] (q7) at (12,-2) {$q_7$};
\path[->] (q2) edge[loop above] node[align=center]{$a, \varepsilon \to a$} (q2);
\path[->] (q3) edge[loop above] node[align=center]{$b$, $a \to \varepsilon$} (q3);
\path[->] (q4) edge[loop above] node[align=center]{$c$, $\varepsilon \to \varepsilon$} (q4);
\path[->] (q5) edge[loop below] node[align=center]{$b$, $\varepsilon \to \varepsilon$} (q5);
\path[->] (q6) edge[loop below] node[align=center]{$c$, $a \to \varepsilon$} (q6);
\path[->] (q1) edge node {$\varepsilon$, $\varepsilon \to \$$} (q2);
\path[->] (q2) edge node[below]{$\varepsilon$, $\varepsilon \to \varepsilon$} (q3);
\path[->] (q2) edge node {$\varepsilon$, $\varepsilon \to \varepsilon$} (q5);
\path[->] (q3) edge node {$\varepsilon$, $\$ \to \varepsilon$} (q4);
\path[->] (q5) edge node {$\varepsilon$, $\varepsilon \to \varepsilon$} (q6);
\path[->] (q6) edge node {$\varepsilon$, $\$ \to \varepsilon$} (q7);
\end{tikzpicture}
\end{center}  

Procure verificar com alguns exemplos que: 
\begin{displaymath}
L(G) = \{a^ib^jc^k : i = j \textrm{ ou } i = k\}  
\end{displaymath}

\end{example}



\section{LLCs são Reconhecíveis por APs}
\label{sec:llc-ap}

Para esta seção usaremos uma abreviação para descrever o empilhamento de uma sequência de símbolos.
Seja $\omega \in \Gamma^*$, $r, q \in Q$, $a \in \Sigma$ e $s \in \Gamma$, escrevemos $\langle r, \omega \rangle \in \Delta(q, a, s)$ para indicar que ao ler $a$ no estado $q$, desempilhamos $s$ e empilhamos cada um dos símbolos de $\omega$ antes de ir para $r$.
Ou seja, se $\omega = s_1 s_2 \dots s_n \in \Gamma^*$, então $\langle r, \omega \rangle \in \Delta(q, a, s)$ é uma abreviação para:

\begin{eqnarray*}
\langle q_1, s_n \rangle & \in & \Delta(q, a, s)\\
\{\langle q_2, s_{n-1} \rangle \} & = & \Delta(q_1, \varepsilon, \varepsilon)\\
\{\langle q_3, s_{n-2} \rangle \} & = & \Delta(q_2, \varepsilon, \varepsilon)\\
&\dots&\\
\{\langle r, s_1 \rangle \} & = & \Delta(q_{n-1}, \varepsilon, \varepsilon)
\end{eqnarray*}

No diagrama de estados, escrevemos:

\begin{center}
\begin{tikzpicture}[node distance=2cm,auto,>=latex]
\tikzset{initial text={}}
\node[state] (q) {$q$};
\node[state] (r) at (6,0) {$s$};
\path[->] (q) edge node {$a$, $s \to s_1 s_2 \dots s_n$} (r);
\end{tikzpicture}
\end{center}

Para abreviar o seguinte:

\begin{center}
\begin{tikzpicture}[node distance=2cm,auto,>=latex]
\tikzset{initial text={}}
\node[state] (q) {$q$};
\node[state] (q1) at (3,0) {$q_1$};
\node[state] (q2) at (6,0) {$q_2$};
\node[state] (qn) at (9,0) {$q_{n-1}$};
\node[state] (r) at (12,0) {$r$};
\path[->] (q) edge node {\small $a$, $s \to s_n$} (q1);
\path[->] (q1) edge node {\small $\varepsilon$, $\varepsilon \to s_{n-1}$} (q2);
\path[dotted] (q2) edge node {} (qn);
\path[->] (qn) edge node {\small $\varepsilon$, $\varepsilon \to s_1$} (r);
\end{tikzpicture}
\end{center}


\begin{example}
\begin{center}
\begin{tikzpicture}[node distance=2cm,auto,>=latex]
\tikzset{initial text={}}
\node[state] (q) {$q$};
\node[state] (r) at (6,0) {$s$};
\path[->] (q) edge node {$a$, $s \to xyz$} (r);
\end{tikzpicture}
\end{center}

É uma abreviação de: 

\begin{center}
\begin{tikzpicture}[node distance=2cm,auto,>=latex]
\tikzset{initial text={}}
\node[state] (q) {$q$};
\node[state] (q1) at (3,0) {$q_1$};
\node[state] (q2) at (6,0) {$q_2$};
\node[state] (r) at (9,0) {$r$};
\path[->] (q) edge node {\small $a$, $s \to z$} (q1);
\path[->] (q1) edge node {\small $\varepsilon$, $\varepsilon \to y$} (q2);
\path[->] (q2) edge node {\small $\varepsilon$, $\varepsilon \to x$} (r);
\end{tikzpicture}
\end{center}
  
\end{example}


\begin{theorem}
  Toda linguagem livre de contexto é reconhecida por um Autômato com Pilha.
\end{theorem}
\begin{proof}
  Se $A$ é uma LLC, por definição, existe uma GLC $G = \langle V, \Sigma, R, S \rangle$ associada a $A$ i.e. $L(G) = A$.

  Construiremos um AP $P = \langle Q, \Sigma, \Gamma, \Delta, q_0, F \rangle$ que reconhece $A$ i.e. $L(P) = A$.

  \begin{itemize}
  \item $Q = \{q_0, q_I, q_F\} \cup E$ onde $E$ é o conjunto de estados necessários para abreviação que vimos acima.
  \item $F = \{q_F\}$
  \item $\Gamma = V \cup \Sigma \cup \{\$\}$
  \item $\Delta$ é apresentado abaixo.
  \end{itemize}


  \begin{eqnarray*}
    \Delta(q_0, \varepsilon, \varepsilon) & = & \{\langle q_I, S\$ \rangle\}\\
    \Delta(q_I, \varepsilon, \$) & = & \{\langle q_F, \varepsilon \rangle\}\\  
    \Delta(q_I, a, a) & = & \{\langle q_I, \varepsilon \rangle\} \textrm{ para todo } a \in \Sigma\\
    \Delta(q_I, \varepsilon, A) & = & \{\langle q_I, \omega \rangle\} \textrm{ para todo } A \to \omega \in R\\
  \end{eqnarray*}

  Diagramaticamente temos:

  \begin{center}
    \begin{tikzpicture}[node distance=2cm,auto,>=latex]
      \tikzset{initial text={}}
      \node[state, initial] (q0) {$q_0$};
      \node[state] (qI) at (3,0) {$q_I$};
      \node[state, accepting] (qF) at (6,0) {$q_F$};
      \path[->] (qI) edge[loop above] node[align=center]{$a, a \to \varepsilon$\\ $\varepsilon, A \to \omega$} (qI);
      \path[->] (q0) edge node {\small $\varepsilon$, $\varepsilon \to S\$$} (qI);
      \path[->] (qI) edge node {\small $\varepsilon$, $\$ \to \varepsilon$} (qF);
    \end{tikzpicture}
  \end{center}

Em palavras, primeiro inserimos $\$$ para marcar o fim da pilha e em seguida inserimos a variável inicial $S$ na pilha e seguimos para o estado intermediário $q_I$.
Então, não-deterministicamente empilhamos o corpo de alguma das regras $A \to \omega$ ou desempilhamos um símbolo terminal $a$ e o reconhecemos.
Quando a pilha chega no fim (no símbolo $\$$), desempilhamos e vamos para o estado final.  
\end{proof}


\begin{example}
  Considere a gramática $G = \langle V, \Sigma, R, S \rangle$ aonde $R$ possui as seguintes regras:

\begin{displaymath}
  S \to 0S1 | \varepsilon
\end{displaymath}

Como já vimos, $L(G) = \{0^n 1^n: n \geq 0\}$.

Usando a construção do teorema anterior, temos que o seguinte AP reconhece essa linguagem:

  \begin{center}
    \begin{tikzpicture}[node distance=2cm,auto,>=latex]
      \tikzset{initial text={}}
      \node[state, initial] (q0) {$q_0$};
      \node[state] (qI) at (3,0) {$q_I$};
      \node[state, accepting] (qF) at (6,0) {$q_F$};
      \path[->] (qI) edge[loop above] node[align=center] {$0, 0 \to \varepsilon$\\ $1, 1 \to \varepsilon$\\ $\varepsilon, S \to \varepsilon$\\ $\varepsilon, S \to 0S1$} (qI);
      \path[->] (q0) edge node {\small $\varepsilon$, $\varepsilon \to S\$$} (qI);
      \path[->] (qI) edge node {\small $\varepsilon$, $\$ \to \varepsilon$} (qF);
    \end{tikzpicture}
  \end{center}

O diagrama acima é uma abreviação para o seguinte diagrama:

  \begin{center}
    \begin{tikzpicture}[node distance=2cm,auto,>=latex]
      \tikzset{initial text={}}
      \node[state, initial] (q0) {$q_0$};
      \node[state] (q1) at (3,0) {$q_1$};
      \node[state] (qI) at (6,0) {$q_I$};
      \node[state] (q2) at (4,-2) {$q_2$};
      \node[state] (q3) at (7,-3) {$q_3$};
      \node[state, accepting] (qF) at (9,0) {$q_F$};
      \path[->] (qI) edge[loop above] node[align=center] {$0, 0 \to \varepsilon$\\ $1, 1 \to \varepsilon$\\ $\varepsilon, S \to \varepsilon$} (qI);
      \path[->] (q0) edge node {\small $\varepsilon$, $\varepsilon \to \$$} (q1);
      \path[->] (q1) edge node {\small $\varepsilon$, $\varepsilon \to S$} (qI);
      \path[->] (qI) edge node[above]{\small $\varepsilon$, $S \to 1$} (q2);
      \path[->] (q2) edge node[below]{\small $\varepsilon$, $\varepsilon \to S$} (q3);
      \path[->] (q3) edge node[above]{\small $\varepsilon$, $\varepsilon \to 0$} (qI);
      \path[->] (qI) edge node {\small $\varepsilon$, $\$ \to \varepsilon$} (qF);
      \path[->] (qI) edge node {\small $\varepsilon$, $\$ \to \varepsilon$} (qF);
    \end{tikzpicture}
  \end{center}
\end{example}


\begin{example}
  
\end{example}


\begin{example}
  
\end{example}


\section{Linguagens Reconhecíveis por APs são Livres e Contexto}
\label{sec:ap-llc}


\begin{lemma}
  Todo AP $P$ é equivalente a outro AP $P'$ em que:
\begin{enumerate}
\item o conjunto de estados finais possui um único elemento $q_f$,
\item as transações só empilham ou desempilham, mas nunca ambas ao mesmo tempo e
\item chega ao estado final com a pilha vazia.
\end{enumerate}
\end{lemma}

\begin{proof}
  Para garantir os itens 1 e 3 criamos transições de cada estado final de $P$ para $q_f'$ que não leem nada e não empilham nem desempilham nada (setas com etiqueta $\varepsilon, \varepsilon \to \varepsilon$).
Uma transição de $q_f'$ para si mesmo que não le nada e desempilha $s$ para cada $s \in \Gamma$ e uma transição que não lê nada e não mexe na pilha que vai de $q_f'$ para $q_f$.


   \begin{center}
    \begin{tikzpicture}[node distance=2cm,auto,>=latex,initial text=,color=blue]
      \tikzset{
        coil/.style={ decorate, decoration={ snake=coil} },
      }
      \node[state, initial] (1) {$q_1$};
      \node[state, accepting, minimum size=0pt] (2) at (3,1){};
      \node[state, accepting, minimum size=0pt] (3) at (3,0){};
      \node[state, accepting, minimum size=0pt] (4) at (3,-1){};
      \path[->] (1) edge[coil]  (2);
      \path[->] (1) edge[coil]  (3);
      \path[->] (1) edge[coil]  (4);
    \end{tikzpicture}
  \end{center}

    \begin{center}
    \begin{tikzpicture}[node distance=2cm,auto,>=latex,initial text=]
      \tikzset{
        coil/.style={ decorate, decoration={ snake=coil} },
      }
      \node[state, initial, color=blue] (1) {$q_1$};
      \node[state, minimum size=0pt, color=blue] (2) at (3,1){};
      \node[state, minimum size=0pt, color=blue] (3) at (3,0){};
      \node[state, minimum size=0pt, color=blue] (4) at (3,-1){};
      \node[state] (f1) at (6,0) {$q_f'$};
      \node[state, accepting] (f) at (9,0) {$q_f$};
      \path[->] (f1) edge[loop above] node {$\varepsilon, s \to \varepsilon$} (f1);
      \path[->, color=blue] (1) edge[coil]  (2);
      \path[->, color=blue] (1) edge[coil]  (3);
      \path[->, color=blue] (1) edge[coil]  (4);
      \path[->] (2) edge[above] node {$\varepsilon, \varepsilon \to \varepsilon$}  (f1);
      \path[->] (3) edge node {$\varepsilon, \varepsilon \to \varepsilon$}  (f1);
      \path[->] (4) edge[below] node {$\varepsilon, \varepsilon \to \varepsilon$}  (f1);
      \path[->] (f1) edge node {$\varepsilon, \varepsilon \to \varepsilon$}  (f);
    \end{tikzpicture}
  \end{center}



Para garantir a condição 2, substituimos toda transição que empilha e desempilha ao mesmo tempo por uma que desempilha seguida por outra que empilha.

\begin{center}
    \begin{tikzpicture}[node distance=2cm,auto,>=latex,initial text=]
      \node[state] (qi) {$q_i$};
      \node[state] (qj) at (3, 0){$q_j$};
      \path[->] (qi) edge node {$a, s_1 \to s_2$} (qj);
    \end{tikzpicture}
  \end{center}

\begin{center}
    \begin{tikzpicture}[node distance=2cm,auto,>=latex,initial text=]
      \node[state] (qi) {$q_i$};
      \node[state] (qk) at (3, 0){$q_k$};
      \node[state] (qj) at (6, 0){$q_j$};
      \path[->] (qi) edge node {$a, s_1 \to \varepsilon$} (qk);
      \path[->] (qk) edge node {$a, \varepsilon \to s_2$} (qj);
    \end{tikzpicture}
  \end{center}
\end{proof}


\begin{theorem}
  Toda linguagem reconhecida por APs é livre de contexto.
\end{theorem}
\begin{proof}
  Seja $P = \langle Q, \Sigma, \Gamma, \Delta, q_0, F \rangle$.
  Pelo lema anterior existe $P'$ equivalente a $P$ satisfazendo as três proriedades.
Criaremos uma gramática $G = \langle \Sigma, V, R, S \rangle$ que reconhece $L(P) = L(P')$.

\begin{itemize}
\item $V = \{A_{pq} : p, q \in Q\}$
\item $S = A_{p_0p_f}$
\item $R$ é formado por três tipos de regras:
\begin{enumerate}
\item $A_{pq} \to A_{pr} A_{rq} \in R$ para todo $p,r,q \in Q$
\item se $\langle r, t \rangle \in \Delta(p, a, \varepsilon)$ e $\langle q, \varepsilon \rangle \in \Delta(s, b, t)$ então $A_{pq} \to a A_{rs} b \in R$
\item $A_{pp} \to \varepsilon \in R$ para todo $p \in Q$
\end{enumerate}
\end{itemize}

Primeiro demonstraremos por indução no tamanho da derivação o seguinte:

{\bf Hipótese de Indução:} Se $A_{pq} \Rightarrow^k x$ então $P$ começa no estado $p$, reconhece $x$ e chega no estado $q$ com a pilha vazia.

{\bf Base:} As únicas derivações de tamanho 1 são da forma $A_{pp} \to \varepsilon$ e claro que o autômato que começa e termina em $q$ reconhece $\varepsilon$.

{\bf Passo de Indução:} Precisamos mostrar que se $A_{pq} \Rightarrow^{k+1} x$ então $P$ começa em $p$, reconhece $x$ e chega em $q$ com a pilha vazia.
O primeiro passo dessa derivação deve ser
\begin{enumerate}
\item $A_{pq} \Rightarrow a A_{rs} b$ ou
\item $A_{pq} \Rightarrow A_{pr} A_{rq}$
\end{enumerate}

No primeiro caso temos que $x = ayb$ e, portanto, $A_{rs} \Rightarrow^k y$.
Pela H.I. $P$ reconhece $y$ indo de $r$ até $s$ e terminando com a pilha vazia.
Como a $A_{pq} \to a A_{rs} b \in R$ então $\langle r, t \rangle \in \Delta(p, a, \varepsilon)$ e $\langle q, \varepsilon \rangle \in \Delta(s, b, t)$.
Então $p$ vai para $r$ e empilha $t$ ao ler $a$ e desempilha $t$ ao ler $b$ e ir para $q$.

No segundo caso, temos que $x = yz$ e $A_{pr} \Rightarrow^* y$ e $A_{rq} \Rightarrow z$.
Ambas derivações devem ter comprimento menor que $k + 1$ e, logo, pela H.I. $P$ reconhece $y$ indo de $p$ para $r$ e reconhece $z$ indo de $r$ até $q$.

Por fim, resta provar por indução no número de passos de computação de $P$ que:

{\bf Hipótese de Indução:} Se $P$ reconhece $x$ indo de $p$ para $q$ em $k$ passos então $A_{pq} \Rightarrow^* x$

{\bf Base:} Em $0$ passos não sai do estado $p$ e reconhece $\varepsilon$.
Pela regra $A_{pp} \to \varepsilon$ geramos $\varepsilon$.

{\bf Passo de Indução:} Suponha que $P$ vai de $p$ até $q$ em $k+1$ passos e reconhece $x$.

Suponhamos primeiro que em nenhum momento no processo a pilha fique vazia.
Neste caso, o símbolo $t$ empilhado no começo é desempilhado no fim.
Se $a$ é o símbolo lido no começo, $b$ o símbolo lido no fim, então $r$ o estado seguinte a $p$ e $s$ é o anterior a $q$.
Ou seja, se $\langle r,t \rangle \in \Delta(p, a, \varepsilon)$ e $\langle q, \varepsilon \rangle \in \Delta(s, b, t)$ então $A_{pq} \to a A_{rs} b \in R$.
Seja $x = ayb$, pela H.I., $A_{rs} \Rightarrow^* y$ e logo $A_{pq} \Rightarrow^* x$.

Por outro lado, se a pilha chega esvaziar entçai ela reconhece uma string $y$ até ficar vazia e $z$ até $q$ e $x = yz$.
Seja $r$ o estado em $P$ quando a pilha está vazia.
Pela H.I. $A_{pr} \Rightarrow^* y$ e $A_{rp} \Rightarrow^* z$.
Como $A_{pq} \to A_{pr} A_{rq} \in R$ então $A_{pq} \Rightarrow^* yz = x$.  
\end{proof}


\begin{corollary}
  Toda linguagem regular é livre de contexto.
\end{corollary}
\begin{proof}
  Não é difícil notar que todo AFD é um AP aonde a pilha nunca é usada.
  Vimos que toda linguagem regular é reconhecida por um AFD, portanto toda linguagem regular é reconhecida por um AP e portanto é livre de contexto.2
\end{proof}

\begin{example}
    \begin{center}
    \begin{tikzpicture}[node distance=2cm,auto,>=latex,initial text=]
      \node[state, initial] (0) {$0$};
      \node[state] (1) at (3,0){$1$};
      \node[state] (2) at (3,-2){$2$};
      \node[state, accepting] (f) at (0,-2){$f$};
      \path[->] (1) edge[loop above] node{$0, \varepsilon \to 0$} (1);
      \path[->] (2) edge[loop below] node{$1, 0 \to \varepsilon$} (2);
      \path[->] (0) edge node{$\varepsilon, \varepsilon \to \$$} (1);
      \path[->] (1) edge node{$1, 0 \to \varepsilon$}  (2);
      \path[->] (2) edge node{$\varepsilon, \$ \to \varepsilon$} (f);
      \path[->] (0) edge node{$\varepsilon, \varepsilon \to \varepsilon$} (f);
    \end{tikzpicture}
  \end{center}  

\begin{displaymath}
  \left.\begin{array}{ccc}
    \langle 1, 0 \rangle & \in & \Delta(1, 0, \varepsilon)\\
    \langle 2, \varepsilon \rangle & \in & \Delta(2, 1, 0)\\
  \end{array}\right\} A_{12} \to 0 A_{12} 1
\end{displaymath}

\begin{displaymath}
  \left.\begin{array}{ccc}
    \langle 2, 0 \rangle & \in & \Delta(1, 0, \varepsilon)\\
    \langle 2, \varepsilon \rangle & \in & \Delta(2, 1, 0)\\
  \end{array}\right\} A_{12} \to 0 A_{22} 1
\end{displaymath}

\begin{displaymath}
  \left.\begin{array}{ccc}
      \langle 1, \$ \rangle & \in & \Delta(0, \varepsilon, \varepsilon)\\
      \langle f, \varepsilon \rangle & \in & \Delta(2, \varepsilon, \$)\\
    \end{array}\right\} A_{0f} \to \varepsilon A_{12} \varepsilon
\end{displaymath}

$P$ reconhece $0011$, portanto essa string deve estar em $L(G)$ onde $G$ é a gramática acima.


\begin{eqnarray*}
  A_{0f} & \Rightarrow & \varepsilon A_{12} \varepsilon \\
        & \Rightarrow & 0 A_{12} 1 \\
        & \Rightarrow & 00 A_{12} 11 \\
        & \Rightarrow & 00 \varepsilon 11 = 0011
\end{eqnarray*}
\end{example}



\section{Linguagens que não são Livres de Contexto}
\label{sec:lnlc}


\begin{lemma}[Bombeamento para LLCs]
  Se $A$ é uma LLC então existe $p$ (comprimento do bombeamento) tal que se $\omega \in A$ e $|\omega| \geq p$ então $\omega = uvxyz$ e:
  \begin{enumerate}
  \item $uv^ixy^iz \in A$ para todo $i \geq 0$
  \item $|vy| > 0$ e
  \item $|vxy| \leq p$
  \end{enumerate}
\end{lemma}

\begin{proof}
  Se $A$ é uma LLC, então por definição existe uma GLC $G = \langle \Sigma, V, R, S \rangle$ tal que $L(G) = A$.
  Seja $b$ o número máximo de símbolos a direita em uma regra em $R$.
  Se partirmos de uma varíavel qualquer em $G$, em $h$ passos o comprimento máximo da string que é possível produzir é $b^h$ (se desenharmos a árvore sintática da string produzida desta forma, $h$ é a altura desta árvore).
  
  O comprimento do bombeamento será $p = b^{|V| + 1}$.
  Se $\omega \in A$ e $|\omega| \geq p$, como na hipótese, existe $k$ tal que $S \Rightarrow^k \omega$. 
  Vamos supor que $k$ seja o menor valor em que $S$ deriva $\omega$.
  Note que necessariamente $k \geq |V| + 1$.
  É claro que esse caminho possui $|V|+1$ símbolos não-terminais, logo, pelo princípio da casa dos pombos, pelo menos uma variável ocorre mais de uma de uma vez neste caminho.
  Seja $T \in V$ a última variável que ocorre mais de uma vez.
  
  Dividimos $\omega$ em $5$ partes $\omega = uvxyz$ de forma que a penúltima ocorrência de $T$ gera $vxy$ e a última gera $x$ (i.e. $S \Rightarrow^* uTz \Rightarrow^* uvTyz \Rightarrow uvxyz$).

%DIAGRAMA

Note que $S \Rightarrow^* uxz$ se substituirmos a penúltima ocorrência de $T$ pela última (i.e. $S \Rightarrow^* uTz \Rightarrow^* uxz$).

%DIAGRAMA

Da mesma forma, $S \Rightarrow^* uv^ixy^iz$ para qualquer $i > 1$ bastando repetir $i$ vezes a última ocorrência de $T$ pela penúltima.

%DIAGRAMA

Se $|vy| = 0$ então $v = y = \varepsilon$ e, portanto, $S \Rightarrow^l uxz = \omega$ como substituindo a penúltima ocorrência de $T$ pela última e $l < k$ contrariando a suposição.

Se $|vxy| > p$ então, pelo princípio da casa dos pombos, na derivação da penúltima ocorrência de $T$ até $vxy$ alguma variável deveria repetir contrariando a suposição de que $T$ era a última variável que se repetia.
\end{proof}


\begin{example}
  $B = \{a^nb^nc^n : n \geq 0\}$ não é livre de contexto.

  Seja $p$ o comprimento do bombeamento, e $\omega = a^pb^pc^p \in B$.
  Se $B$ fosse uma LLC então, pelo lema, $\omega = uvxyz$, $|vy| > 0$ e $uv^ixy^iz \in B$ para todo $i \geq 0$.
  Temos duas possibilidades:
  \begin{enumerate}
  \item se $v$ e $y$ um único tipo de símbolo cada, então $uv^2xy^2z$ não conterá a mesma quantidade de $a$s, $b$s e $c$s e, portanto, $uv^2xy^2z \notin B$.
  \item se $v$ ou $y$ contém mais de um símbolo distinto então $uv^2xy^2z$ contém símbolos na ordem errada e, portanto, $uv^2xy^2z \notin B$. 
  \end{enumerate}
  Concluímos que $B$ não é livre de contexto.
\end{example}

\begin{example}
  $C = \{a^ib^jc^k : 0 \leq i \leq j \leq k\}$ não é livre de contexto.

  Seja $p$ o comprimento do bombeamento e $\omega = a^pb^pc^p \in C$.
  Pelo lema, se $C$ fosse livre de contexto, teríamos $\omega = uvxyz$ com $|vy| > 0$ e $uv^ixy^iz \in C$ para todo $i > 0$.
  Considere os dois possíveis casos:

  \begin{enumerate}
  \item $y$ e $v$ só contém um tipo de símbolo cada:
    \begin{itemize}
    \item se $a$ não ocorre em $vy$ então $uxz \notin C$;
    \item se $b$ não ocorre em $vy$, mas $a$ ocorre, então $uv^2xy^2z \notin C$, pois possui mais $a$s do que $b$s e se $c$ ocorre $uv^2xy^2z \notin C$ por motivo análogo;
    \item se $c$ não ocorre em $vy$ então $uv^2xy^2z \notin C$, pois a string possuiria mais $a$s ou mais $b$s do que $c$.
    \end{itemize}
  \item se $y$ ou $z$ possuem mais de um tipo de símbolo então $uv^2xy^2z \notin C$, pois possui símbolos na ordem errada.
  \end{enumerate}
  Concluímos que $C$ não é livre de contexto.
\end{example}

Na seção anterior, vimos que todas as linguagens regulares são livres de contexto, mas anteriormente havíamos mostrado que exitem linguagens livres de contexto ($\{0^n1^n: n \geq 0\}$ por exemplo) que não são regulares.

\begin{displaymath}
  \textrm{Ling. Reg.} \subset \textrm{LLCs}
\end{displaymath}

Nesta seção vimos que esxistem linguagens formais que não são livres de contexto:

\begin{displaymath}
  \textrm{LLCs} \subset \textrm{Ling. Formais}
\end{displaymath}

Além disso, vimos que linguagens regulares coincidem com as linguagens reconhecíveis por autômatos finitos e que as livres de contexto coincidem com as reconhecíveis por autômatos com pilha.

No próximo capítulo passaremos à questão central do curso, a saber, a existência de problemas que não possuem solução computacional.
Para tanto, precisamos de um modelo de computação capaz de dar conta de qualquer dispositivo mecânico.

